\chapter{Introduction}

The intent of \verb`clang-immutability-check` is to be the first pass for
ensuring logically immutability for all classes in a C++ project.
We only focus on classes, ensuring logically immutability across all sequences
of method calls.
The scope of the first pass is limited; for example it does not attempt to
understand pointers.
It will only check simple methods as valid, mainly any which just return a field
or a field with some arithmetic applied to it.
For these simple method checks it also ensures that the return type is
transitively \const{}.
For instance, if a class contains a vector of pointers to a class \verb`C` and
returns that field, we would ensure the return type is
\verb`const vector<const C*>&` instead of \verb`const vector<C *>&`.

The secondary purpose of this work is to have a database of information which
helps with the LLVM analysis and stores the results of this analysis.
In addition we have website which acts both as a documentation hub and a way to
present the results.
This website lists all the classes we analyze and shows the results for each
method we encounter.
Using the database we can tabulate our results to answer the following research
questions:

\begin{enumerate}
  \item How many classes contain \const{} methods.
  \item How many \const{} vs non-\const{} methods are there.
  \item How many \const{} methods are ``obviously'' \const{}
\end{enumerate}

From there we'll have a better sense of exactly where the complexity of more
difficult \const{} problems are.

These complex methods are then analyzed by our heavyweight LLVM static analysis.
The information from the first stage such as type qualifiers and other
information usually not accessible in LLVM is available through our database
allowing a more precise analysis than just LLVM itself.
